
%-----------------------------------------------------------------------------%
%								   Keywords							  	      %
%-----------------------------------------------------------------------------%

% PURPOSE: Keywords will allow employers to quickly find students that are interested in the fields they are looking for.

% INSTRUCTIONS: Add the pertinent keywords from the list below to the line that begins with "\noindent"
%               Please only use the keywords from the list below. 

% E.g. If you are interested in appliying data science to specific applications in academia the command would look 
%      something like:

% \noindent Keywords: data science, applications, academia

% List of possible keywords (Please take ONLY the keywords from this list, do not add your own):
% computational modeling
% data science
% algorithms
% applications
% industry
% national lab
% academia
% non-profit

% Uncomment packages to test compilation 
% \documentclass[11pt]{article} 
% \usepackage[includefoot, includehead, left=1in, right=1in, top=1in, bottom=0.25in]{geometry}
% \usepackage{graphicx}
% \usepackage{epsfig}
% \usepackage{amsmath}
% \usepackage{subfigure} 
% \usepackage{hyperref}
% \usepackage{amssymb}
% \usepackage{amsmath}
% \usepackage{latexsym}
% \usepackage{color}
% \usepackage[compact]{titlesec}
% \usepackage[usenames,dvipsnames,svgnames,table]{xcolor}
% \usepackage{hyperref}
% \usepackage[all]{hypcap}    
% \hypersetup{
%     colorlinks,
%     citecolor=blue,
%     filecolor=blue,
%     linkcolor=blue,
%     urlcolor=blue
% }

% Uncomment to test in your own latex editor
% \begin{document} 


%-----------------------------------------------------------------------------%
%								   Cover Letter						  	      %
%-----------------------------------------------------------------------------%


% INSTRUCTIONS: Please edit the below letter to reflect your research/internship/job goals. Below is a template to act 
% 				as an example but feel free to structure it how you see fit. 
%
% IMPORTANT: Keep this cover letter under 1 page. If it is not, your merge request will be rejected. When compiled,
%			 the page settings will be 1 in margins on all sides except for the  bottom which will be a 0.25 in margin
%            The font will be 11 pt font. You can test this in online latex editors like overleaf or on your own latex 
%			 editor to guage the size of the compiled pdf.
%
%            Be sure to uncomment the packages above and the begin and end document commands so that it compiles. 

% Add your keywords here
\noindent {\textbf{Keywords:}} data science, algorithms, industry


\vspace{2em} % space between keywords and cover letter


% Add your name here
\section{Student Name}

% Also add your keywords to this part. Just copy and paste them after "Keywords:"
\addtocontents{toc}{Keywords: data science, algorithms, industry\par}

% add 'firstname_lastname' with no upper case letters
\label{sec:student_b} 


\setlength\parindent{0cm}

% Edit this letter to make it as you wish. Keep in mind this page must be no longer than 1 page compiled. 
To whom it may concern:\\

My name is (name), and I am a doctoral student in the Department of CMSE with an anticipated graduation date of (date). {\textbf{I am interested in summer internship opportunities in industry or the national laboratory system, focusing generally in areas of computational modeling.}}\\

Prior to joining the CMSE PhD program in Fall 2017, I received a B.S. in (subject) at (school) [and additional education as relevant].  My current research focuses on (short explanation), and I am generally interested in research in (more areas).  I am interested in pursuing an internship to explore career paths outside of academia and to gain new technical and “soft” skills (or other reasons as relevant), and would be interested in pursuing opportunities in all areas of computational modeling.\\

My resume is included with this letter, and I can be contacted at (contact info).\\

\begin{flushright}
Sincerely, \\
\vspace{1em} 
\vspace{1em} 
First Last Name\\
\end{flushright}




% Uncomment to test in your own latex editor
% \end{document} 

