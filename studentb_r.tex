
%---------------------------------------------------------------------------------------------------------------------%
%                                                   Short Resume                                                      %
%---------------------------------------------------------------------------------------------------------------------%

% INSTRUCTIONS: Please fill out the below template with the most pertinent and recent work for each section. The
%               purpose of this resume is to give employers a very brief overview of what different students are
%               working on in CMSE and to allow them to select potential candidates for internships, jobs, etc. 
%               The idea is that after reviewing this document, they will request more information on students 
%               that seem like a good fit for what they are looking for so there is no need to be verbose here.
%               Just a short and susinct description of what you are doing and what you are looking for in an 
%               intenship/job.
%
% IMPORTANT: This document must be one page when compiled, test it in your own latex editor of choice. 
%            Be sure to uncomment the packages and the begin and end document commands so that it compiles. 

% Uncomment packages to test compilation 
% \documentclass[11pt]{article} 
% \usepackage[includefoot, includehead, left=1in, right=1in, top=1in, bottom=0.25in]{geometry}
% \usepackage{graphicx}
% \usepackage{epsfig}
% \usepackage{amsmath}
% \usepackage{subfigure} 
% \usepackage{hyperref}
% \usepackage{amssymb}
% \usepackage{amsmath}
% \usepackage{latexsym}
% \usepackage{color}
% \usepackage[compact]{titlesec}
% \usepackage[usenames,dvipsnames,svgnames,table]{xcolor}
% \usepackage{hyperref}
% \usepackage[all]{hypcap}    
% \hypersetup{
%     colorlinks,
%     citecolor=blue,
%     filecolor=blue,
%     linkcolor=blue,
%     urlcolor=blue
% }

% Uncomment to test in your own latex editor
%\begin{document} 

\noindent  \LARGE{\textbf{Student B}} % Add name here

% \hline 
\normalsize
%%%%%%%%%%%%%%%%% CONTACT INFORMATION %%%%%%%%%%%%%%%%%
% Your email address, phone number 
\noindent \href{mailto:email@msu.edu}{email@msu.edu}

\noindent\hrulefill
\vspace{1em}

%%%%%%%%%%%%%%%%% MAIN BODY %%%%%%%%%%%%%%%%%%%%%%%%%%%
% The main body is contained in a tabular environment. To move sections onto the next page, simply end the tabular environment and begin a new tabular environment.
\noindent \begin{tabular}{@{} l l}

    \Large{Focus} & My focus is in blah...\\
    & \\
    \Large{Education}    & \textbf{Michigan State University} \\
    & Ph.D., Computational Mathematics, Science, and Engineering, May 2022. \\
    & \\
    & \textbf{University} \\
    & B.S., Major, Year. \\
    & \\
    \Large{Experience}    & {\parbox{4.3in}{ \vspace{1.2em} Short description/summary of research and estimation techniques.}}\\
    & \\
    \Large{Presentations}   & \textbf{Posters} \\
    & Poster Title, Location, Year\\
    & \\
    &\textbf{Talks} \\
    & Talk Title, Location, Year\\
    & \\
    \Large{Projects}   & \textbf{Project 1}: \\
    & {\parbox{4.3in}{Short description/summary of pretinent project.}}\\
    & \\
    \Large{Relevant Coursework }    & \\
    & Course Name, Semester (Fall, Spring, Summer) Year \\
    & \\
    \Large{Programming}   & Python, MATLAB, C/C++, etc. \\
    \Large{Languages}& \\
    &\\
    \Large{Software}    & \LaTeX, Mathematica  \\
    &\\
    \Large{Awards and }    & \textbf{Graduate Student Teacher of the Year, Department} \\
    \Large{Fellowships}   & Course Name, Year \\
    & \\
    & \textbf{X Felowship} \\
    & City, Country, Year \\
    & \\
    \Large{Citzenship} & United States

\end{tabular}

% Uncomment to test in your own latex editor
%\end{document}
